\vspace{-0.75em}
\section{Conclusion}
\vspace{-0.75em}

Contrary to common belief, global file system namespaces can be scalable if
they are given enough domain-specific knowledge.  We show that many of today's
applications are specialized, so they have regular, large namespaces. As a
result, the file system should be changing its internal mechanisms to leverage
the bounded and balanced nature of these namespaces to optimize metadata
performance.  Namespace schemas and generators solve many file system mtadata
read problems because clients and servers avoid exchanging large lists. Our
examples benefit from  \emph{metadata compaction}, which speeds up
network/storage overheads and gives clients and servers the ability to
\emph{modify large namespaces} and \emph{generate relevant parts of the
namespace}.

%File systems are thought to
%be robust and general because they have been around for a long time. But we
%show that today's applications are specialized, so they have regular, large
%namespaces. As a result, the file system should be changing its internal
%mechanisms to leverage the bounded and balanced nature of these namespaces to
%optimize metadata performance.

%In this section, we show how clients and metadata servers communicate using the
%Pattern PLFS language and present our
%storage system that adapts to the wokload
%(Section~\ref{sec:adapting-to-the-workload-with-cudele})).  Other destructive
%solutions include changing the storage system and altering the application.
%
%\subsection{Adapting to the Workload with Cudele}
%\label{sec:adapting-to-the-workload-with-cudele}
%
%\begin{figure}[tb]
%\centering
%  \includegraphics[width=90mm]{figures/arch.png} 
%  \caption{System XX lets clients optimize performance by telling the storage
%  system about the workload. Clients can specify a Structured Namespace (blue
%  subtrees and Section~\ref{sec:structured-namespaces}) or by merging file system
%  metadata from an Unstructured Namespace (red subtree and
%  Section~\ref{sec:unstructured-namespaces}).}\label{fig:arch}
%\end{figure}
%
%% What is Cudele
%Cudele is a file system with programmable consistency and durability. Clients
%use an API to decouple existing subtrees from the global namespace; metadata
%operations from the other clients targeted at the decoupled subtree can be
%programmed to be blocked or marked as overwritable. With the decoupled subtree
%in hand, the client can do metadata operations locally. Upon completion, the
%client can merge the subtree back into the global namespace. 
%
%% Why Cudele is a good fit for implied namespaces
%Cudele has the mechanisms for understanding the file system metadata language
%and adapting to the workload.  Figure~\ref{fig:arch} shows how clients decouple
%the namespace with the Cudele API, specifying how many extra inodes they want
%and the structure for the namespace they intend to create. The metadata server
%and client both know about the metadata in the blue subtree, requiring no RPCs,
%and if the client creates more metadata (red subtree), it can merge it back
%into the global namespace.  This model lets users enjoy the simplicity of
%global namespaces and the high performance of node-local operations.  We extend
%the API to support the declaration of structured namespaces and leverage the
%existing API to merge unstructured namespaces. 
%
%\subsubsection{Structured Namespaces}
%\label{sec:structured-namespaces}
%
%% What is a structured namespace
%A structured namespace is created according to a pattern. If both the client
%and metadata server knows the pattern, they can create the metadata
%independently. This has two benefits: (1) it reduces RPCs which improves
%performance and reduces network traffic and (2) it allows the client and server
%to operate in parallel.  The patterns that Cudele understands are shown in
%Listing~\ref{src:example} and the programmable interfaces are shown below.
%There are two parameters for unstructured namespaces: \texttt{pattern} and
%\texttt{trigger}. 
%
%\subsubsection{Trigger: Start Namespace Construction}
%
%% How does trigger work and why do we neet it
%\texttt{trigger} specifies when to start the namespace construction on the
%metadata server.  The metadata reconstruction can be asynchronous and saving
%this resource intense process for later can have better performance. To
%facilitate the exploration of different trigger policies, we make the value for
%the \texttt{trigger} parameter programmable.  Administrators inject Lua code
%that specifies or calculates thresholds for when to start namespace
%construction. Although we make this programmable, we do not make any
%conclusions about the best trigger time and leave the exploration of this space
%as future work.
%
%% example
%In Listing~\ref{src:example}, the trigger is:
%\begin{listing}
%\begin{minted}[frame=single,
%               framesep=2mm,
%               xleftmargin=10pt,
%               tabsize=2]{lua}
%{
%  if MDSs[whoami]["cpu"] > 30
%}
%\end{minted}
%\label{src:thresh}
%\end{listing}
%
%which means that construction of the namespace will start if current MDS
%(\texttt{whoami}) has a CPU utilization (\texttt{``cpu"}) above 30\%.
%
%% Drawbacks: consistency
%Triggering construction asynchronously can improve performance because the
%process can be deferred until the system has less load. However, this
%performance gain comes at the cost of consistency. Even if the construction is
%triggered immediately, the metadata is eventually consistent; other clients see
%outdated metadata because the namespace is sitting on the client. Delaying the
%trigger improves the liklihood that system finds a window of low load but also
%increases the latency of other clients.\\
%
%\noindent\emph{Implementation}: we re-use the polling and embedded Lua
%virtual machine in Mantle~\cite{sevilla:sc15-mantle} to implement the trigger
%interface. By default, every 10 seconds the metadata server checks if the
%condition for triggering is satisfied by executing the Lua code. Mantle has
%variables exposed for administrators to explore load balancing policies; just
%like this work, some of these policies need to identify overloaded metadata
%servers so we re-use all those variables.  Some of the more useful variables
%include:
%
%\begin{itemize}
%  \item Memory Usage
%  \item CPU Utilization
%  \item Request Rate
%  \item Queue Depth
%  \item Server Tags: whoami, i
%\end{itemize}
%
%\subsubsection{Pattern: Express Namespace}
%\label{sec:pattern-express-namespace}
%
%% How does pattern work and why do we need it
%\texttt{pattern} describes the metadata layout of the Structured Namespaces. It
%is the same language used in~\cite{he:hpdc13-plfs-patterns}. When the metadata
%server starts a namespace construction, it creates all the file system metadata
%generated by this formula. As a refresher, the pattern in Listing~\ref{src:example}:
%
%  \[[i, (d[0], d[1], ...)^r]\]
%
%means that there are \(r\) entries in the PLFS index file, where the first
%entry has a physical offset of \(i\) and lengths of \(d\), where the pattern in
%\(d\) repeats. \\
%
%% WTF -- this doesn't give file system metadata! ARGGGGG is it file creations
%% or index files shit?
%
%% Drawbacks
%
%\noindent\emph{Implementation}: Another big fat TODO.
%
%\begin{listing}
%\begin{minted}[frame=single,
%               framesep=2mm,
%               xleftmargin=10pt,
%               tabsize=2]{js}
%{
%  <!-- Structured Namespace Pattern !-->
%  "S_pattern": "[i, (d[0], d[1], ...)^r]",
%  
%  <!-- Structured Namespace Trigger !-->
%  "S_trigger": "if MDSs[whoami]["cpu"] > 30",
%  
%  <!-- Untructured Namespace Allocated Inos !-->
%  "US_alloci": "1000",
%}
%\end{minted}
%\caption{Using the Cudele API to express metadata structure, which is
%understood by both the server and client.}
%\label{src:example}
%\end{listing}
%
%\subsubsection{Unstructured Namespaces}
%\label{sec:unstructured-namespaces}
%
%\subsubsection{Migrating Metadata Construction}
%\label{sec:migrating-metadata-construction}
%
