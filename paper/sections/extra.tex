\subsubsection{Language: Pattern PLFS}
\label{sec:language-patterned-io}

% What other approaches or solutions existed at the time that this work was done?
I/O access patterns are studied extensively and results are integrated into
existing systems. The common checkpointing technique, employed by ADIOS and
PLFS, transform the concurrently written file into exclusively written file
fragments. 

% What was wrong with the other approaches or solutions?
Despite extensive studies on I/O access patterns, current systems do not
dynamically recognize patterns at a fine granularity. Because the PLFS
checkpoint technique makes many small writes, it is either slow (on disk) or
consumes a large amount of space (memory).  

% What is the authors' approach or solution?
The authors present algorithms to discover and replace PLFS metadata. The
system is composed of: 

\begin{itemize}

  \item local per-process metadata: split based on pattern discovering engine
  (get tuples using sliding window)

  \item merge local indices into a single global one per PLFS (check if local
  neighbors abut each other)

\end{itemize}

% Why is it better than the other approaches or solutions?
The authors' algorithms rediscover information as data moves through POSIX. By
dynamically  pattern matching and compression, they are able to reduce latency
and disk/memory usage on reads. 

% How does it perform?
They tested with FS-TEST, MapReplayer, and real applications. In their
experiments, metadata is reduced by several orders of magnitude, write
performance is increased (by 40\%), and reads are increased (by 480\%). 

% Why is this work important?

Discovering structure in unstructured IO is useful for other systems, like
pre-fetching and pre-allocation of blocks in file system or SciHadoop
(metadata/data). This work that these algorithms (applied to compress metadata)
can successfully optimize I/O. 

% 3+ comments/questions
\begin{itemize}

  \item What PLFS structures allow us to to this?

  \item How dependent on workloads are these?

  \item Can this be extended to other file systems?

\end{itemize}


